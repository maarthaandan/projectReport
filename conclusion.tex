\chapter{Conclusion}
It is clear from the results obtained that the super resolved image obtained from Generative Adversarial Network is close to reality. It has a higher PSNR score than the older methods like Bicubic Interpolation, but the score is lower compared to that obtained with SRResNet. But the image obtained from GAN is perceptually closer to reality and thus would entail a higher Mean Opinion Score (MOS).
The output obtained is constrained by the effectiveness of the training procedure. Training for the generator along was done with NVIDIA Jetson TX1. But training the Generative Adversarial Network (GAN) as a whole required a machine of higher computing power. The GAN was trained using an NVIDIA Quadro M5000 GPU. Another important parameter in the training procedure is the dataset. The richer the dataset, the network learns with different types of images and thus will be able to super resolve any kind of input image.
