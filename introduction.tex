\chapter{Introduction}
Image processing is an interesting field with a lot of areas of research. This includes recognizing the contents of an image, manipulating an image, increasing the resolution of an image, etc. Image Super Resolution is an extensive area of research within itself. A lot of research has gone into solving the problem of Single Image Super Resolution which has resulted in methods with better accuracy and speed. Even then, one problem remains largely unsolved. The problem of recovering the finer texture details at large upscaling factors. The choice of the objective function plays a major role in the behaviour of optimization based super resolution methods. The previous works have mainly focussed on minimizing the Mean Squared Error Loss (MSE Loss). Minimization of MSE Loss gives solutions with high signal to noise ratio but are seen lacking in high frequency details. Generative Adversarial Networks can be used for Image Super Resolution to produce photorealistic images for 4X upscaling factors.   

The report has been divided into six chapters. The second chapter gives an insight into the background information needed in the understanding of the report.The third chapter handles the literature survey done as part of the project. Several papers were referred during the course of completion of this project. They are spoken about in the third chapter. The fourth chapter essentially deals with the procedure. That is, the several steps involved in completion of the project. The results obtained are summarized and analyzed in the fifth chapter. The report concludes with the sixth chapter. 